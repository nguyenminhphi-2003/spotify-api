\chapter{Giới thiệu đề tài}

\section{Lý do chọn đề tài}

Trong kỷ nguyên công nghệ số phát triển mạnh mẽ như hiện nay, âm nhạc trực tuyến không chỉ là một hình thức giải trí phổ biến mà còn là một phần thiết yếu trong đời sống tinh thần của con người. Với sự bùng nổ của các nền tảng âm nhạc trực tuyến như Spotify, Apple Music, hay Zing MP3, người dùng ngày càng quen thuộc với việc nghe nhạc mọi lúc, mọi nơi, trên mọi thiết bị. Những nền tảng này không chỉ cung cấp thư viện nhạc khổng lồ, mà còn tích hợp các công nghệ tiên tiến như trí tuệ nhân tạo (AI) để cá nhân hoá trải nghiệm người nghe.

Chính vì vậy, nhóm quyết định lựa chọn đề tài xây dựng hệ thống API mô phỏng nền tảng nghe nhạc trực tuyến – lấy cảm hứng từ Spotify – nhằm khám phá và ứng dụng những công nghệ hiện đại trong việc xây dựng một hệ thống backend thực tế. Đề tài này không chỉ giúp nhóm nâng cao kiến thức về thiết kế hệ thống, mà còn tạo điều kiện để rèn luyện các kỹ năng mềm trong môi trường làm việc nhóm.

Cụ thể, đề tài hướng đến các mục tiêu sau:
\begin{itemize}
    \item Tìm hiểu và áp dụng các công nghệ hiện đại như Django REST Framework, MongoDB, Postman, Docker, v.v.
    \item Thiết kế và phát triển một hệ thống API có khả năng mô phỏng các chức năng cơ bản của một ứng dụng nghe nhạc: đăng ký, đăng nhập, phát nhạc, tạo và quản lý danh sách phát.
    \item Rèn luyện kỹ năng lập trình backend, thiết kế cơ sở dữ liệu NoSQL và triển khai API theo kiến trúc RESTful.
    \item Nâng cao tinh thần làm việc nhóm, khả năng tổ chức và quản lý dự án, đặc biệt là trong môi trường mô phỏng thực tế.
\end{itemize}

\section{Bối cảnh thực hiện}

Đề tài được thực hiện trong khuôn khổ môn học \textbf{Hệ điều hành}, thuộc chương trình đào tạo ngành Công nghệ Thông tin tại \textbf{Trường Đại học Sài Gòn}. Trong học phần này, sinh viên được yêu cầu áp dụng tổng hợp các kiến thức lý thuyết và kỹ năng lập trình đã học để xây dựng một hệ thống có tính ứng dụng thực tế.

Với mong muốn kết nối lý thuyết và thực hành, nhóm quyết định chọn đề tài xây dựng hệ thống backend API phục vụ một ứng dụng âm nhạc trực tuyến, xem đây như một bước đệm để tiếp cận các dự án thực tế trong tương lai.

\begin{itemize}
    \item \textbf{Thời gian thực hiện}: Trong học kỳ II năm học 2024–2025.
    \item \textbf{Địa điểm}: Phòng thực hành Công nghệ Thông tin, Trường Đại học Sài Gòn.
    \item \textbf{Công cụ sử dụng}: Django REST Framework, MongoDB, Postman, GitHub, Docker, VS Code.
    \item \textbf{Phạm vi}: Tập trung phát triển hệ thống backend API ổn định, bảo mật và sẵn sàng mở rộng; frontend sẽ được đề cập trong giai đoạn sau.
\end{itemize}

\section{Mục tiêu của đề tài}

Đề tài hướng đến việc xây dựng một hệ thống backend mạnh mẽ, có khả năng mô phỏng trải nghiệm cơ bản của người dùng trên nền tảng âm nhạc trực tuyến. Các mục tiêu cụ thể bao gồm:

\begin{itemize}
    \item Xây dựng hệ thống RESTful API phục vụ các chức năng: xác thực người dùng, phát nhạc, quản lý danh sách phát.
    \item Thiết kế cơ sở dữ liệu sử dụng MongoDB, đảm bảo dữ liệu được tổ chức hợp lý và dễ truy vấn.
    \item Bảo mật hệ thống bằng cách tích hợp JWT (JSON Web Token) trong quá trình xác thực người dùng.
    \item Viết tài liệu hướng dẫn sử dụng API, giúp các lập trình viên frontend có thể tích hợp hệ thống một cách thuận tiện.
    \item Đảm bảo tính mở rộng và dễ bảo trì của hệ thống, hướng đến khả năng phát triển lâu dài.
\end{itemize}

\section{Ý nghĩa thực tiễn}

Đề tài mang lại nhiều giá trị thực tiễn, không chỉ ở khía cạnh học thuật mà còn cả kỹ năng nghề nghiệp:

\begin{itemize}
    \item Giúp sinh viên tiếp cận quy trình phát triển một hệ thống backend API thực tế, từ khâu thiết kế, lập trình đến triển khai.
    \item Là bước đệm quan trọng cho các dự án lớn hơn trong tương lai, đặc biệt là các hệ thống dịch vụ số liên quan đến âm nhạc, video hoặc nội dung số khác.
    \item Tài liệu và mã nguồn của đề tài có thể trở thành nguồn tham khảo hữu ích cho các bạn sinh viên khác trong các học phần liên quan đến lập trình backend, phát triển API hoặc quản lý dữ liệu NoSQL.
    \item Nâng cao kỹ năng giải quyết vấn đề, tư duy logic, làm việc nhóm và tinh thần tự học – những yếu tố quan trọng trong nghề nghiệp lập trình.
\end{itemize}
